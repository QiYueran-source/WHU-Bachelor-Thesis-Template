% Chapter 3

\chapter{研究设计}
计算基于多Agent的预期协同度因子AED涉及到多方面工作,主要包括数据、模型和系统设计。\par
数据是训练的基石,没有合适的数据和特征工程,即便模型再精妙,也无法发挥作用。\par 
模型是计算的核心,对于不同的AI-Agent,需要引入模型异质性,模拟不同投资者的差异。\par 
系统设计是实现的关键,由于多Agent协同工作的需求,必须设计合理的系统架构,才能保证计算的高效性。\par 

\section{数据} 

\subsection{数据来源}

\subsection{数据处理}

\subsection{数据分析}


\section{模型设计}

\section{系统设计}
\subsection{架构设计}

\subsubsection{设计动机}

本研究涉及全市场、长时序、多因子的数据规模与多Agent并行的训练需求,若采用单机全量加载与集中训练,将面临内存与算力瓶颈,因此采用基于主从节点模式的分布式架构,将数据管理与训练执行分离,以控制单机资源占用并支持水平扩展。\par

\textbf{全量数据内存估算。} 因子数据规模为:证券数 $N=3500$、年份 $Y=28$、每月一条记录故月数 $T=12$、因子数 $F=88$。全量因子矩阵的浮点规模为 $N \times Y \times T \times F = 3500 \times 28 \times 12 \times 88 = 1.035 \times 10^8$ 个数值;按双精度(8字节)存储约 $828$ MB,考虑 DataFrame 索引、列名及预处理过程中的中间副本(如标准化、合并收益率等),单机全量常驻内存约 $1.5 \sim 2.5$ GB。若进一步保留多期滚动窗口或多份快照,内存易达 $3$ GB 以上,对普通开发机或单台训练节点不友好。\par

\textbf{训练阶段内存与算力估算。} 训练时显存与 CPU 占用与输入形式、批量大小和网络结构相关。(1)\textbf{显存}:单证券输入(如形状 $12 \times 88$)、小批量(如 256)下,MLP 等小模型约 $0.5 \sim 2$ GB 显存;若采用多月多证券张量(如 $12 \times 3500 \times 88$)或大批量,仅单批输入即达数百 MB 至数 GB,加上模型参数、梯度和优化器状态,总显存需求可达数 GB 至十余 GB。(2)\textbf{CPU}:数据线程负责从 Redis 拉取数据片、预处理与组 batch,多线程时 CPU 占用与 worker 数、序列长度成正比;消息与网络 I/O 在中央节点与训练节点间会带来额外 CPU 与带宽消耗,但相比全量加载与大规模矩阵运算仍更可控。\par

因此,通过中央节点按需加载与按“年—月—证券”切片供给、训练节点按滚动窗口拉取数据并执行训练,可避免单机全量载入,将单节点内存控制在可接受范围内,并便于通过增加训练节点扩展算力。

\subsubsection{整体架构}

系统采用主从节点模式:中央节点负责数据管理、消息调度与结果汇总,多个训练节点通过 Redis 与消息总线从中央获取数据并执行 Agent 训练。为了实现节点通信和节点端口分配,需要公网服务器提供frp服务,如图~\ref{fig:architecture} 所示。

\begin{figure}[htbp]
  \centering
  \includegraphics[width=0.95\textwidth]{figures/architecture.png}
  \caption{整体架构:中央节点与训练节点}
  \label{fig:architecture}
\end{figure}

中央节点部署于具备数据库与 Redis 访问权限的主机,承担四类职责:(1)从数据库按需加载因子与收益率数据,并且完成预处理;(2)通过消息总线与数据流水线将数据按“年—月—证券”切片写入 Redis,供训练节点按滚动窗口拉取;(3)对训练任务进行调度与监控;(4)接收并汇总各训练节点上报的指标与模型信息,进行实证分析。训练节点可多台并列,每节点在独立容器内运行:从 Redis 拉取数据片,经过适配后执行训练,将结果与模型写回中央或推送至统一仓库。中央与节点之间通过frp隧道通信,实现数据与算力分离、训练节点水平扩展。  

\subsection{中央节点}
\subsubsection{数据组织}  
\subsubsection{数据流水线}    

\subsection{训练节点} 
\subsubsection{容器}  
\subsubsection{工作流}  
\paragraph{数据线程}
\paragraph{训练线程}
\subsubsection{结果收集}

