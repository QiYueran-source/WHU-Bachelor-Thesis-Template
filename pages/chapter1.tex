% Chapter 1

\chapter{引言}
在当下的中国股票市场中,散户群体是交易活动的绝对主体,贡献了超80\%的市场交易量,
散户投资者究竟能否通过学习上市公司的基本信息来形成理性预期、平抑市场非理性波动,学界对此长期存在分歧。
其预期偏误对市场价格的非理性波动具有不可忽视的放大作用。
由于个人投资者的信息处理能力相对有限,面对日益庞杂的市场数据,其预期的形成路径也变得愈加多元而难以把握。
另一方面,机构投资者尽管具备较强的信息解读能力,却在处理同质信息时易于步调一致,
进而诱导散户产生跟风行为,反而可能加剧价格的剧烈震荡。
对此,近年来的重要政策文件——包括2023年中央金融工作会议及党的二十届四中全会——均明确将引导市场预期、维护资本市场预期稳定列为重要议题。
在此背景下,深入探讨投资者如何基于企业多维特征形成主观预期、进而影响资产定价,对于推进我国金融市场的预期管理实践具有重要的理论价值与现实意义。